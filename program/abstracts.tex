%!TEX TS-program = xelatex
%!TEX encoding = UTF-8 Unicode

% use the corresponding paper size for your ticket definition
\documentclass[letterpaper,12pt]{article}

%%% Load fonts and graphics
\usepackage{mathspec} % loads fontspec as well
\usepackage{xcolor,xifthen,xltxtra,xunicode,graphicx,amstext}
\definecolor{RegentGrey}{HTML}{83939D}
\usepackage[pdfauthor={Testing Gravity 2017},pdftitle={Talks and Posters},colorlinks,urlcolor={RegentGrey}]{hyperref}
\defaultfontfeatures{Scale=MatchLowercase,Ligatures=TeX}

%%% Set paper size and margins
\usepackage[letterpaper]{anysize}       % Set paper size and margins
\marginsize{0.5in}{0.5in}{0.5in}{0.5in}
\setlength{\headheight}{32pt}
\setlength{\headsep}{12pt}
\flushbottom

%%% Customize layout
\usepackage{fancyhdr}
\pagestyle{fancy}
\pagestyle{fancy}
\lhead{\fontspec{Cinzel}\huge Testing Gravity 2017}\chead{}
\rhead{\fontspec{Lato Light Italic}\Large Talks and Posters, page~\thepage}
\lfoot{}\cfoot{}\rfoot{}
%\renewcommand{\headrulewidth}{1pt}
%\renewcommand{\footrulewidth}{1pt}


\setmainfont[BoldFont={Lato},BoldItalicFont={Lato Italic}]{Lato Light}
\setsansfont{Lato}
\setmonofont{Jura}
\setmathsfont(Digits,Latin,Greek)[Scale=MatchLowercase]{Lato Light}

\newcommand{\slot}[1]{\item[\fontspec{Lato} #1]}
\newcommand{\talk}[2]{{\fontspec{Lato Bold} #1,} {\fontspec{Lato Italic} #2}}


\begin{document}
\begin{itemize}
\setlength\itemsep{0pt}
%\setlength\itemindent{36pt}

\item \talk{Hartmut Abele (TUWien)}{TBA}

\item \talk{Eric Adelberger (University of Washington)}{TBA}

\item \talk{Niayesh Afshordi (University of Waterloo and Perimeter Institute)}{Echoes from the abyss}

I will present tentative evidence for Planck-scale structure near black hole
horizons, from LIGO data.

\item \talk{Bruce Allen (MPI \& LIGO)}{Direct observation of gravitational waves from the merger and inspiral of two black holes}

This talk describes the LIGO observations of gravitational waves emitted by the
final few orbits and merger of two black holes. I present our main results, as well as
some of the ``behind the scenes'' details of the 14 September 2015 discovery, and briefly
describe our 26 December 2015 detection of a somewhat weaker gravitational waves from a
similar system. The consistency of these observations with the predictions of General
Relativity, and our expectations for the second observing run O2 (which began at the end
of 2016) and for future observing runs are also discussed.

References:\\
B.~P.~Abbott et al.,
{\em Phys.\ Rev.\ Lett.}\ {\bf 116}, 061102 (2016);
{\em Phys.\ Rev.\ Lett.}\ {\bf 116}, 241103 (2016);
{\em Phys.\ Rev.\ Lett.}\ {\bf 116}, 221101 (2016).

\item \talk{Kamal Barghout (Prince Mohammad Bin Fahd University)}{Modification to Newtonian gravity of two types of opposite masses}

The model assigns Coulombic gravitational interaction to dark matter and
baryonic particles, where like masses repel and unlike masses attract. It provides a
physical explanation to MOND theory and explains flat rotation curves for spiral galaxies.

\item \talk{Masha Baryakhtar (Perimeter Institute)}{Searching for ultralight particles with black holes and gravitational waves}

\item \talk{Joel Berg\'e (ONERA)}{MICROSCOPE on its way to the tightest test of the weak equivalence principle}

MICROSCOPE is a CNES/ONERA mission that aims to test the Weak Equivalence
Principle in space with an unprecedented accuracy (100 times higher than that of current
tests). The satellite was launched on April 25, 2016, and its commissioning phase has been
completed in November 2016. MICROSCOPE is now starting its science operations. In this
talk, I will first present MICROSCOPE science goals before introducing its experimental
concept. I will then report on the outcome on the commissioning phase.

\item \talk{Supranta S. Boruah (University of Waterloo)}{Cuscuton bounce --- a novel bouncing scenario}

In this work, we study the perturbation theory in a universe with Cuscuton. In
particular, we show that we can have a stable bouncing cosmology in this theory without
deadly instabilities.

\item \talk{Cliff Burgess (McMaster University and Perimeter Institute)}{Effective field theories and modifying gravity: The view from below}

We live at a time of contradictory messages about how successfully we
understand gravity. General Relativity seems to work very well in the
Earth's immediate neighbourhood, but arguments abound that it needs
modification at very small and/or very large distances. This talk tries
to put this discussion into the broader context of similar situations in
other areas of physics, and summarizes some of the lessons which our
good understanding of gravity in the solar system has for proponents for
its modification over very long and very short distances. The main
message is that effective theories (in the technical sense of effective)
provide the natural (and arguably only known) precise language for
framing proposals. Its framework is also useful, inasmuch as it makes
some modifications seem more plausible than others, though there are
also some potential surprises.

\item \talk{Clare Burrage (University of Nottingham)}{Experimental searches for screened dark energy}

\item \talk{Pasquale Bosso (University of Lethbridge)}{Potential tests of quantum gravity}

Most quantum gravity theories predict the Generalized Uncertainty Principle
(GUP) to replace the Heisenberg principle near the Planck scale. We show that GUP in turn
predicts potentially observable quantum gravity effects in quantum optical systems.

\item \talk{Phillipe Brax (Commisariat a l'Energie Atomique-Saclay)}{Gravitational birefringence}

We will present new results on the propagation and emission of gravitational
waves in bigravity.

\item \talk{Ilaria Caiazzo (University of British Columbia)}{Testing gravity with X-ray polarization from accreting black holes}

X-ray polarimetry will open a new observational window on black holes. It will
provide us with information on the geometry of the emission region with unprecedented
resolution, allowing us to probe strong gravity in black holes' accretion disks.

\item \talk{Cedric Deffayet (IAP, Paris)}{Partial masslessness beyond Einstein}

We discuss the theory of a partially massless graviton on non Einsteinian spacetimes.
\item \talk {Claudia de Rham (Imperial College)}{Binary pulsar tests of gravity}

\item \talk {Guillem Domenech (Yukawa Institute for Theoretical Physics)}{Tensor modes produce a scale dependent non-Gaussianity}

Planck 2015 data shows a possible scale dependent local non-Gaussianity. I will
show how such feature is naturally achieved if tensor modes significantly contribute to
primordial NG. This is possible in a space-time symmetry breaking theory.

\item \talk{Daniel D'Orazio (Harvard University)}{Tools for characterizing a massive black hole binary population}

Characterization of a massive black hole binary (MBHB) population, through
electromagnetic (EM) signatures, will inform expectations for the low-frequency
gravitational wave sky. I will discuss novel predictions for such EM signatures of MBHBs.

\item \talk{Sheila E. Dwyer (LIGO Hanford Observatory and Caltech)}{Prospects for gravitational wave observations: Update from LIGO}

Since making the first measurements of gravitational waves in fall of 2016, LIGO
scientists have improved the performance of the LIGO detectors. LIGO Hanford observatory
has focused on high power operation, while LIGO Livingston Observatory has prima 

\item \talk{Matteo Fasiello (Standford)}{LSS probes for dark energy \& modified gravity?}

The perturbative treatment of LSS, especially in its effective theory
realization, can well describe the quasi-linear regime. I will elaborate on recent
progress towards including a dark energy or a modified gravity component in this
framework.

\item \talk{Pedro Ferreira (Oxford Univesity)}{A complete framework for testing gravity on cosmological scales}

We will be able to constrain gravity with linear perturbations on cosmological
scales with the next generation of surveys. I will discuss various aspects of such an
endeavour: the role of theoretical priors, the importance of the extra degrees of freedom,
the play off between large and smaller (quasi-static scales) and the impact of non-linear
evolution and screening. I will show that the future is promising and that constraints
could be comparable to those we can obtain on Solar System scales.

\item \talk{Jose Galvez (Simon Fraser University)}{Nowhere to hide: Unscreening chameleons with a black hole}

It is believed that the additional degrees of freedom in any modification of
gravity are screened by the matter densities coexisting in an astrophysical Black Hole. In
this talk, we will find that this might not always be the case.

\item \talk{Kouichirou Horiguchi (Nagoya University)}{Proofs of the cosmological phase transitions}

It is expected that the universe had experienced many kinds of phase
transitions. These phase transitions are difficult to observe directly, but their relics,
called ``cosmic defects'' can be a smoking gun of them. We have investigated several types
of observables related to cosmological perturbations that the cosmic defects should induce, 
namely, primordial magnetic fields, gravitational waves, gravitational lensing and so on. By 
considering near-future observations, we have found that cosmic strings and texture with
$G\mu \sim Gv^2 \sim 10^{-7}$ may be detected by SKA and DECIGO with $G$, $\mu$ and 
$v$ being the gravitational constant, the tension of the strings and the field value of the texture, respectively.

\item \talk{Lam Hui (Columbia University)}{Light boson dark matter and conference summary}

\item \talk{Mustapha Ishak (The University of Texas at Dallas)}{Cosmological consistency tests of gravity theory and cosmic acceleration}

Testing general relativity at cosmological scales and probing the cause of
cosmic acceleration are among the important objectives targeted by incoming and future
astronomical surveys and experiments. I present our recent results on consistency tests.

\item \talk{Henri Inchauspé (ONERA)}{The inverse square law and Newtonian dynamics space experiment (ISLAND) mission concept}

An electrostatic torsion pendulum embedded in a gradiometer, onboard a drag-free
spacecraft bound to the outer Solar System, testing gravity both at submilliter scales and
at large scales in an optimal environment.

\item \talk{Bhuvnesh Jain (University of Pennsylvania)}{Tests of gravity and dark matter interactions with halos and voids}

\item \talk{Austin Joyce (The University of Chicago)}{TBA}

\item \talk{Vitali Halenka (University of Michigan)}{Testing gravity with gravitational potentials of galaxy clusters}

The research involves several topics such as designing new tests to discriminate
between general relativity and modified gravity and to constraint standard cosmological
model.

\item \talk{Justin Khoury (University of Pennsylvania)}{Cosmic acceleration without dark energy}

\item \talk{Tim Kovachy (Stanford University)}{Testing the equivalence principle with macroscopic atom interferometers}

Matter wave interferometers that cover macroscopic scales in space and in time
have a high intrinsic sensitivity to inertial forces, making them a promising
candidate for a wide range of precision measurement applications. We show how to overcome
the technical barriers to the experimental realization of these interferometers, and we
demonstrate the application of these interferometers to dual species acceleration
measurements for a test of the weak equivalence principle and to precision gravity
gradiometry. The dual species measurements make use of simultaneous, co-located atom
interferometers using Rb-85 and Rb-87. I will present preliminary data from these
measurements and discuss progress toward an equivalence principle test. Additionally, I
will discuss the observation of the phase shift associated with spacetime curvature across
a particle's wavefunction, commonly called the quantum curvature phase shift. The
quantum curvature phase shift arises because spacetime curvature induces tidal forces on
the wavefunction of a single quantum system.

\item \talk{David Langlois (APC, Paris)}{Degenerate higher order scalar tensor theories beyond Horndeski}

I will present scalar tensor theories whose Lagrangian contains second order
derivatives of a scalar field and which propagate only one scalar mode, in addition to the
two tensor modes. These theories encompass and extend the so-called Horndeski theories. I
will discuss some phenomenological aspects of these new theories.

\item \talk{Andrei G. Lebed (University of Arizona)}{Inequality between inertial and gravitational masses: Suggested experiment on the Earth's orbit}

We have theoretically demonstrated that passive gravitational mass of a
composite quantum body is not equivalent to its inertial mass. We have suggested an
idealized experiment on the Earth's orbit to observe this phenomenon both for the theory
and experiment. Here, we discuss how to perform the corresponding
real experiment. It would be not only the first observation of a breakdown of the
Equivalence Principle, but also the first observation of quantum effects in General
Relativity.

\item \talk{Jounghun Lee (Seoul National University)}{A bound violation on the galaxy group scale: The turn-around radius of NGC 5353/4}

The first observational evidence for the violation of the maximum turn-around
radius on the galaxy group scale is presented and its implication on the nature of gravity
will be discussed.

\item \talk{Danielle Leonard (Carnegie Mellon University)}{Testing gravity with $E_G$: Mapping theory onto observation}

Upcoming cosmological surveys will offer an unprecedented opportunity to test
gravity on large scales. One method to do so is via the statistic $E_G$, which combines weak
lensing, galaxy clustering, and redshift-space distortions. In order to optimally test
gravity with $E_G$, we require a robust understanding of its theoretical value. I will
discuss theoretical uncertainties affecting $E_G$, and implications for testing gravity with
this statistic.

\item \talk{Yin-Zhe Ma (University of KwaZulu-Natal)}{Constraining gravity and primordial non-Gaussianity with cosmological observations}

In this talk, I will present two aspects of the cosmological observations and
its constraints on fundamental physics: constraining gravity and primordial non-
Gaussianity. The peculiar velocity field is one of the important probes of large scale
structure. We can compare the difference between predicted velocity field with measured
velocity field, and is able to measure the growth of structure at different evolution
epoch. This gives strong constrains on the evolution of different modify gravity models.
The second method uses the fact that if primordial non-Gaussianity exists on large scales,
there will be extra-correlation between the offsets of peculiar velocity field and linear reconstructed
density field. We therefore use the current peculiar velocity field data and
set up strong constraints on $f_{\text{NL}}$ parameter. In addition, we use 21-cm intensity
mapping technique to constrain the primordial non-Gaussianity, and forecast the
prospective detectability from MeerKAT and SKA radio surveys. This will set up strong
limit on the initial condition of the Universe.

\item \talk{Roy Maartens (University of Cape Town)}{Cosmology on ultra-large scales with the SKA}

\item \talk{Scott Menary (York University)}{The ALPHA-G experiment}
\item \talk{Holger Mueller (University of California Berkeley)}{Using matter waves to look for the dark sector}
\item \talk{Ryo Namba (McGill University)}{Inflationary models driven by vector fields}

Most vector-driven inflationary models are plagued by pathological instabilities or
disfavored by observations. I address existing issues and propose a new class of stable
models in general vector-tensor theories broadening the possible model window.

\item \talk{Alex Nielsen (AEI Max-Planck)}{Testing black hole ringdowns with gravitational waves}
\item \talk{Junpei Ooba (Nagoya University)}{Cosmological constraints on scalar-tensor cosmology and the variation of the gravitational constant}

We present cosmological constraints on the scalar-tensor theory of gravity by
analyzing the angular power spectrum data of the cosmic microwave background (CMB)
obtained from the Planck 2015 results together with the baryon acoustic oscillations (BAO)
data. We find that the inclusion of the BAO data makes more than 10\% improvement on the
constraints on the time variation of the effective gravitational constant. We also discuss
the dependence of the constraints on the choice of the prior.

\item \talk{Manu Paranjape (Universit\'e de Montr\'eal)}{Gravitationally induced quantum transitions and the graviton laser}

We calculate the probability for resonantly induced transitions in quantum
states due to time dependent gravitational perturbations. Contrary to common wisdom, the
probability of inducing transitions is not infinitesimally small. We consider a system of
ultra cold neutrons (UCN), which are organized according to the energy levels of the
Schr\"odinger equation in the presence of the earth's gravitational field. Transitions
between energy levels are induced by an oscillating driving force of frequency $\omega$.
The driving force is created by oscillating a macroscopic mass in the neighbourhood of the
system of neutrons. The neutrons decay in 880 seconds while the probability of
transitions increase as $t^2$. Hence the optimal strategy is to drive the system for 2
lifetimes. The transition amplitude then is of the order of $1.06\times 10^{-5}$ hence
with a million ultra cold neutrons, one should be able to observe transitions. The same
system can be used to think about the possibility of creating a graviton laser. It is
possible to create a population inversion by pumping the system using the phonons. We
compute the rate of spontaneous emission of gravitons and the rate of the subsequent
stimulated emission of gravitons. The gain obtainable is directly proportional to the
density of the lasing medium and the fraction of the population inversion. The
applications of a graviton laser would be interesting.

\item \talk{Will Percival (ICG, Portsmouth)}{Final cosmological measurements from the Baryon Oscillation Spectroscopic Survey (BOSS)}

The Baryon Oscillation Spectroscopic Survey (BOSS), undertaken as part of the
Sloan Digital Sky Survey (SDSS), has obtained redshifts for 1.15 million galaxies covering
a volume of 12.3 Gpc$^3$. The galaxy catalogues, which have just been publicly released,
provide a goldmine of cosmological information. In this seminar I will briefly introduce
the survey and the sample, and present the key cosmological measurements that test large scale
gravity.

\item \talk{Maxim Pospelov (University of Victoria and Perimeter Institute)}{Dark matter}

\item \talk{Frans Pretorius (Princeton University)}{Testing GR and constraining alternative theories with LIGO observations}

\item \talk{David Rapetti (CU Boulder, NASA Ames)}{Modeling and constraining the cluster mass function to test gravity at large scales}

Using a robust self-consistent analysis including survey, observable-mass
scaling relations and weak gravitational lensing data to calibrate the absolute mass
scale, we obtained the most recent constraints on f(R) gravity from the abundance of
massive galaxy clusters, which are an order of magnitude tighter than the best previously
achieved. Based on the current highest resolution N-body simulations, our new modeling of
the f(R) halo mass function includes novel corrections to capture key non-linear effects
of the Chameleon screening mechanism. These results will allow us to obtain the next
generation of cluster constraints on this model, and provide a technique that can also be
applied to other proposed cosmological modifications of gravity.

\item \talk{Janina Renk (Stockholm University)}{Signatures of Horndeski's gravity on ultra-large cosmic scales}

\item \talk{Hiromi Saida (Daido University)}{Observing GR effect and mass of the massive BH at the center of our galaxy: GR calculation and Subaru telescope observation}

In order to measure general relativistic (GR) effects of strong gravity of BH,
we have been studying a star, so-called S2, orbiting the massive BH candidate at the
galactic center (Sgr A*). The gravity that S2 will experience at its pericenter passage in
2018 (about one year after this conference) is about two orders of magnitude stronger than
the so-far electromagnetically observed gravitational field such as the Hulse-Taylor
pulsar. This talk reports that, according to our GR numerical calculation of S2's motion
and our infra-red observations of S2 by Subaru telescope, the BH's GR effect will be
detected with ten sigma precision, and the BH's mass will be measured with one
percent accuracy. We expect these objectives will be achieved by 2020.

\item \talk{Yuki Sakakihara (Kyoto University)}{Cosmology with bigravity theory}

As an infrared modification of gravity, we discuss dRGT bigravity theory,
focusing on if it can consistently explain cosmological evolution both in early time and
in late time. We also discuss fluctuations generated during inflation in bigravity.

\item \talk{Misao Sasaki (Yukawa Institute for Theoretical Physics)}{Signatures from inflationary massive gravity}

\item \talk{John R. Shaw (University of British Columbia)}{Probing dark energy with the Canadian Hydrogen Intensity Mapping Experiment (CHIME)}

CHIME is a transit radio interferometer designed specifically for probing dark
energy. I will discuss its goals, its status and describe the powerful new analysis
techniques we have developed to confront the many challenges of such observations.

\item \talk{Maresuke Shiraishi (Kavli Institute for the Physics and Mathematics of the Universe)}{Testing cosmic parity violation with CMB 2, 3, 4 point-correlators}

Under the existence of chiral sources, such as a vector field coupled to an
axion, primordial correlators break parity. I introduce interesting signatures of induced
CMB 2, 3, 4-point correlators, and their current and future constraints.

\item \talk{Hajime Sotani (National Astronomical Observatory of Japan)}{Gravitational waves from protoneutron stars}

We examine the time evolution of the frequencies of the gravitational wave after
the bounce within the framework of relativistic linear perturbation theory using the
results of one-dimensional numerical simulations of core-collapse supernovae. Protoneutron
star models are constructed in such a way that the mass and the radius of the protoneutron
star become equivalent to the results obtained from the numerical simulations. Then we
find that the frequencies of gravitational waves radiating from protoneutron stars
strongly depend on the mass and the radius of protoneutron stars, but almost independently
of the profiles of the electron fraction and the entropy per baryon inside the star.
Additionally, we find that the frequencies of gravitational waves can be characterized by
the square root of the average density of the protoneutron star irrespective of the
progenitor models, which are completely different from the empirical formula for cold
neutron stars. The dependence of the spectra on the mass and the radius is different from
that of the g-mode: the oscillations around the surface of protoneutron stars due to the
convection and the standing accretion-shock instability. Careful observation of these
modes of gravitational waves can determine the evolution of the mass and the radius of
protoneutron stars after core bounce. Furthermore, the expected frequencies of
gravitational waves are around a few hundred hertz in the early stages after bounce, which
must be a good candidate for the ground-based gravitational wave detectors.

\item \talk{Ingrid Stairs (University of British Columbia)}{Tests of strong-field gravity with pulsars}

\item \talk{Leo Stein (California Institute of Technology)}{Black hole mergers: Beyond general relativity}

One hundred years after the birth of general relativity, advanced LIGO
has finally directly detected gravitational waves. The source: two black holes merging
into one. Advanced LIGO will soon provide the opportunity to test GR, using gravitational
waves, in the dynamical strong-field regime --- a setting where GR has not yet been tested. GR
has passed all weak-field tests with flying colors. Yet it should
eventually break down, so we must look to the strong-field. To perform strong-field tests
of GR, we need waveform models from theories {\em beyond} GR. To date there are no numerical
simulations of black hole mergers in theories which differ from GR. The main obstacle is
the mathematical one of well-posedness. I will explain how to overcome this obstacle, and
demonstrate the success of this approach by presenting the first numerical simulations of
black hole mergers in a theory beyond GR.

\item \talk{Norihiro Tanahashi (Osaka University)}{Causal structure and shock formation in the most general scalar-tensor theories}

Wave propagation speed in scalar-tensor theories may vary depending on
environment, and it can cause many peculiar phenomena such as superluminal propagation and
wave form distortion. In this work, we focus on Horndeski theory and its generalization to
two scalar fields, and examine causal structure and shock formation in those theories.
About the causal structure, we focus on a stationary black hole, and check if scalar or
gravitational wave could come out from black hole interior to exterior by propagating
superluminally. We find that the scalar field must satisfy certain conditions to prohibit
such a propagation across the horizon. About the shock formation, we examine conditions
for that to occur in Horndeski theory. When the background solution is symmetric enough,
we find that the gravitational wave is protected from the shock formation while the scalar
field suffers from it. On less symmetric backgrounds in Horndeski theory, and in more
general theories such as those with multiple scalar fields, the shock formation may occur
even for gravitational wave. We discuss implications of such phenomena.

\item \talk{Jay Tasson (Carleton College)}{Testing local Lorentz invariance with gravitational waves}

Following a review of the framework for testing local Lorentz invariance
provided by the gravitational Standard-Model Extension, applications to gravitational-wave
related searches will be discussed.

\item \talk{David Wenjie Tian (Instituto de Ciencias Nucleares -- Universidad Nacional Autonoma de M\'exico)}{Primordial nucleosynthesis: A revised semi-analytical approach}

\item \talk{Mark Trodden (University of Pennsylvania)}{Multi-messenger time delays from lensed gravitational waves}

\item \talk{Shinji Tsujikawa}{Cosmology in generalized Proca theories}

We study the cosmology in the presence of a massive vector field with derivative
interactions that propagates only the 3 desired polarizations with second-order equations
of motion.

\item \talk{Tanmay Vachaspati (Arizona State University)}{Quantum radiation during gravitational collapse}

\item \talk{Alexander Vikman (Prague)}{Canonical exorcism for cosmological ghosts}

I will discuss canonical ways to extinguish the ghosts around time-dependent backgrounds.

\item \talk{Kent Yagi (Princeton University)}{What do GW150914 and GW151226 tell us about extreme gravity?}

LIGO's discovery of the direct detection of gravitational waves allow us to
probe gravity in extreme gravity for the first time. I will describe how well GW150914 and
GW151226 probe fundamental pillars of General Relativity and current limitations.

\item \talk{Yasuho Yamashita (Yukawa Institute)}{Constraint on ghost-free bigravity from Cherenkov radiation}

We investigate the gravitational Cherenkov radiation (GCR) in the ghost-free
bigravity model. We show that the GCR emitted even from an ultrahigh energy cosmic ray is
sufficiently suppressed for the graviton's effective mass less than 100eV.

\item \talk{Zeeshan Yousaf (University of the Punjab)}{Wormhole solutions in modified gravity}

In this talk, I shall use cut and paste technique in order to construct thinshell
wormhole of a stellar system with f(R) terms. I shall assume f(R) model as a source
of exotic content in the wormhole throat.

\item \talk{Aaron Zimmerman (Canaidian Institute for Theoretical Astrophysics)}{The horizon modes of black holes}

Near horizon sources and perturbations generically produce ``horizon modes''
whose appearance mimics the usual quasinormal modes. The horizon modes emanate from close
to the black hole horizon, and so serve as direct probes of black hole horizons.

\end{itemize}
\end{document}
